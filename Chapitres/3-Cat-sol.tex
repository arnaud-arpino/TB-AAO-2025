\section{Catalogue des solutions et tests effectués}
\subsection{Pré-étude des micro-capsules}

Les micro-capsules sont fabriqué à partir de verre borosilicate, ce type 
de verre est souvent utilisé pour la verrerie de laboratoire. En effet, il comporte une 
très bonne résistance aux chocs thermique du faite de son faible coefficient de dilatation. 
De plus, il est résistant à de nombreux produits chimiques.
Néanmoins, son comportement mécanique peut s'avéré fragile en particulier pour de faibles
épaisseurs (voir \cite{report_borosilicate}).


\subsection{Liste des solutions envisagées}
\subsubsection{Canon à azote}
\paragraph{Principe de la solution}

\paragraph{Tests et simulations}

\paragraph{Conclusion}

\subsubsection{Implosion de la capsule}
\paragraph{Principe de la solution}

\paragraph{Tests et simulations}

\paragraph{Conclusion}


\subsubsection{Actionneur mécanique}
\paragraph{Principe de la solution}

\paragraph{Tests et simulations}

\paragraph{Conclusion}

\subsubsection{Fréquence de résonance}
\paragraph{Principe de la solution}

\paragraph{Tests et simulations}

\paragraph{Conclusion}


\subsection{Critères et choix de la solution}

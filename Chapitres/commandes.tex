%------------- Commandes utiles ----------------

\section{Quelques commandes}

Voici quelques commandes utiles :

%------ Pour insérer et citer une image centralisée -----

\insererfigure{logos/Import.jpg}{3cm}{Légende de la figure}{Label de la figure}
% Le premier argument est le chemin pour la photo
% Le deuxième est la hauteur de la photo
% Le troisième la légende
% Le quatrième le label
Ici, je cite l'image \ref{fig: Label de la figure}


%------- Pour insérer et citer une équation --------------

\begin{equation} \label{eq: exemple}
\rho + \Delta = 42
\end{equation}

L'équation \ref{eq: exemple} est cité ici. 

% ------- Pour écrire des variables ----------------------

Pour écrire des variables dans le texte, il suffit de mettre le symbole \$ entre le texte souhaité comme : constante $\rho$.

Pour utiliser pythontex : (pour les chapitres un fichier tex séparé pour python a été préparé)
\begin{pycode}
x = 3
\end{pycode}
       
$x = \py{x}$

si x = ?? effectuer :
pythontex main.tex


Pour créer une bibliographie avec des citations :

Ceci est du texte avec une citation complète comme note de bas de page.\footfullcite{smith2023}
Ceci est une référence d'un site \footfullcite{Biscactus}
Ceci est une référence d'un site \footfullcite{LaTeX}

\documentclass{rapport}
\usepackage{pythontex}
\usepackage{csquotes}
\addbibresource{library.bib}
\title{Rapport laboratoire de matériaux : nickelage} %Titre du fichier

\begin{document}

%----------- Informations du rapport ---------

\unif{HEIG}
\titre{Conception d'un système automatisé permettant le cassage de microcapsules en verre et la libération contrôlée de réactifs chimiques} %Titre du fichier .pdf
\cours{Travail de bachelor} %Nom du cours
\sujet{Laboratoire Swiss cat +} %Nom du sujet


\enseignant{Giuseppe \textsc{Costanzo}}%Nom de l'enseignant

\eleves{Arnaud \textsc{Arpino} } %Nom des élèves

\entreprise{EPFL - Swiss Cat + \\
Keyan VILLAT\\
Henryk ZOLNOWSKI
}

\nom{Arnaud ARPINO}

%----------- Initialisation -------------------
        
\fairemarges %Afficher les marges
\fairepagedegarde %Créer la page de garde

\pagenumbering{roman} 
\clearpage
\null
\thispagestyle{empty}
\clearpage

\vspace*{4cm}
\section*{\Huge Préambule}
\addcontentsline{toc}{section}{Préambule} % Ajout à la table des matières

\vspace{1cm}
Ce travail de Bachelor (ci-après TB) est réalisé en fin de cursus d'études, en vue de
l'obtention du titre de Bachelor of Science HES-SO en Ingénierie.
En tant que travail académique, son contenu, sans préjuger de sa valeur, n'engage
ni la responsabilité de l'auteur, ni celles du jury du travail de Bachelor et de l'École.
Toute utilisation, même partielle, de ce TB doit être faite dans le respect du droit
d'auteur.

\vspace{2cm}

\hspace*{10cm}
\begin{minipage}{0.3\textwidth}
    HEIG-VD\\
    Le Chef du Département
\end{minipage}

\vspace{5cm}
Yverdon-les-bains, le \today

\clearpage
\null
\thispagestyle{empty}
\clearpage
\newpage

\vspace*{4cm}
\section*{\Huge Authentification}
\addcontentsline{toc}{section}{Authentification}

\vspace{2cm}
Je soussigné, Arnaud Arpino, atteste par la présente avoir réalisé seul ce travail et
n'avoir utilisé aucune autre source que celles expressément mentionnées.

\vspace{2cm}

\hspace*{10cm}
\begin{minipage}{0.3\textwidth}
    Arnaud ARPINO\\
    \includegraphics[height=3cm]{Images/Signatures/signature Arnaud.png}
\end{minipage}

\vspace{5cm}
Yverdon-les-bains, le \today

\newpage

\clearpage
\null
\thispagestyle{empty}
\clearpage
\newpage

\vspace*{1cm}
\section*{\Huge Complémentaire concernant \\
l'utilisation d'outils\\
d'intelligence artificielle}
\addcontentsline{toc}{section}{Complémentaire concernant l'utilisation d'outils d'intelligence artificielle}

\vspace{2cm}

L'utilisation limitée d'outils dits d'intelligence artificielle ou plus particulièrement de
LLM (Large Language Models) a été validée avant le début de ce travail de bachelor
pour les utilisations spécifiques suivantes :

\vspace{1cm}
Utilisation de ChatGPT de l'entreprise OpenAI versions GPT-3, GPT-4 et ChatGPT-4-turbo pour
obtenir rapidement des informations servant de double vérifications, de correction
orthographique, d'aide pour le language LATEX, d'aide pour les langages de programmation python et 
autres langages utiles à la programmation du bras robotisé robot.

\vspace{1cm}
Je soussigné, M. Arnaud Arpino, atteste par la présente avoir nullement utilisé de
logiciels de génération de texte automatique pour la rédaction de ce document et
que toutes les resources spécifiques utilisées se trouvent dans la bibliographie ou en
annexe de ce rapport.

\vspace{2cm}

\hspace*{10cm}
\begin{minipage}{0.3\textwidth}
    Arnaud ARPINO\\
    \includegraphics[height=3cm]{Images/Signatures/signature Arnaud.png}
\end{minipage}

\vspace{1cm}
Yverdon-les-bains, le \today



\newpage

\clearpage
\null
\thispagestyle{empty}
\clearpage
\newpage

\section*{\Huge Résumé}
\addcontentsline{toc}{section}{Résumé}

\clearpage
\null
\thispagestyle{empty}
\clearpage
\newpage 

\clearpage
\pagestyle{plain} 

\thispagestyle{roman}
\tabledematieres %Créer la table de matières
\newpage

\thispagestyle{roman}
\listoftables
\newpage

\thispagestyle{roman}
\listoffigures

\clearpage
\pagenumbering{arabic} 
\pagestyle{fancy}
\setcounter{page}{1}

%------------ Corps du rapport ----------------

\section{Introduction}

\subsection{Contexte}
Dans le cadre de la formation de microtechnicien, l'étudiant doit réaliser un Travail de Bachelor pour valider ses compétences. Ainsi, ce travail se déroule en collaboration avec le laboratoire Swiss Cat + de l'EPFL. Le projet doit être réalisé sur une durée de 420 heures, réparties entre mi février et la fin du mois de juillet.

\vspace{0.3cm}
Le laboratoire Swiss Cat + est "une infrastructure axée sur les données
pour la découverte et l'optimisation des catalyseurs" d'après le site internet \cite{swisscatweb}.

L'objectif principal du laboratoire est l'automatisation robotique à haut débit d'expérimentation dans le domaine de la chimie, combinée à une analyse avancée soutenue par l'intelligence artificielle.

\vspace{0.3cm}
Le projet est subdivisé en deux hubs l'un se concentre sur la catalyse homogène à EPFL et l'autre sur la catalyse hétérogène à l'ETHZ.

\begin{figure}[H]
    \centering
    \includegraphics[width=13cm]{Images/Illustrations/Intro/Labo_swisscat_vue3D.jpg}
    \label{fig:Labo-Vue-3D}
    \caption{Vue d'ensemble du laboratoire Swiss Cat + sur le campus EPFL}
\end{figure}

\subsection{Description du projet}
Le projet du TB, intervient au sein du projet StoRMS, une solution innovant visant à préparer des solutions chimiques de manière automatisé, par la manipulation de micro-capsules de réactifs chimiques solides. 

\vspace{0.3cm}
Les micro-capsules permettent de stocker les différentes quantités de réactif de manière imprécise. Dans un second temps, on mesure leurs masses, puis on combine plusieurs micro-capsules pour obtenir la quantité exacte de réactif nécessaire pour la réaction chimique.

\vspace{0.3cm}
Cependant, ces micro-capsules sont celées lors de leur remplissage, il est donc nécessaire de les ouvrir pour libérer le réactif. C'est à ce moment que le système de destruction de micro-capsules entre en jeu.

\subsection{Organisation}
Le document est divisé comme suit:
\begin{itemize}[label=\textbullet]
    \item \textbf{Introduction}: Présentation du projet et de son contexte.
    \item \textbf{Analyse du besoin}: Identification des besoins du système.
    \item \textbf{Fonctions et exigences du système}: Définition des fonctions de services et techniques du système.
    \item \textbf{Catalogue des solutions}: Présentation des solutions techniques envisagées.
    \item \textbf{Modélisation 3D et réalisation}: Description de la réalisation du système.
    \item \textbf{Tests et validation}: Présentation des tests effectués et des résultats obtenus.
    \item \textbf{Conclusion}: Bilan du projet et perspectives d'amélioration.
\end{itemize}
\newpage
\section{Cahier des charges}

\subsection{Définitions}

\begin{itemize}[label=\textbullet]
    \item \textbf{Micro-capsules}: petit cylindres en verre fermés des deux côtés (borosilicate). \\ 
    Diamètre extérieur = 2,8 ± 0,05 mm; \\
    Diamètre intérieur = 2,5 ± 0,05 mm; \\
    Longueur = 10 mm

    \item \textbf{Réacteurs}: Flacons en verre dimension: 32 x 11.6, 1.5 ml \\ pour plaques « Para-Dox »
    Référence: \cite{ref_vial}

    \item \textbf{Bloc de réaction}: Plaques « Para-Dox » avec 48 positions, Gen II, pour flacons 12 x 32
    
    \item \textbf{Glove-box}: Espace de travail sous atmosphère contrôlée, rempli d'azote à température 
    ambiante (environ 25 °C) et en surpression (+15 Pa par rapport à 1 Atm).
    \item \textbf{Cross contamination}: Contamination entre différents réactifs 
    causée par des restes dans le système d'ouverture ou par projection
\end{itemize}

\subsection{Analyse du besoin}

\begin{figure}[H]
    \centering
    \includegraphics[width=15cm]{Images/Illustrations/CDH/Bete a corne.png}
    \label{fig:beteacorne}
    \caption{Diagramme bête à corne}
\end{figure}

\begin{table}[H]
    \centering
    \begin{tabular}{
    >{\columncolor[HTML]{FFFFFF}}l |
    >{\columncolor[HTML]{FFFFFF}}l }
    {\color[HTML]{000000} \textbf{\#}} & {\color[HTML]{000000} \textbf{Besoin}}         \\ \hline
    {\color[HTML]{000000} \textbf{1}} & {\color[HTML]{000000} Ouvrir des micro-capsules} \\ 
    \end{tabular}
    \caption{Liste des besoins du système}
    \label{tab:besoin}
    \end{table}

\subsection{Fonctions et exigences du système}
\subsubsection{Fonctions de services}

Les fonctions de services correspondes aux exigences principales du produits.

\begin{table}[H]
  \centering
  \begin{tabular}{cl|cl}
  \multicolumn{2}{l|}{\textbf{Fonctions de service}} &
    \multicolumn{2}{l}{\textbf{Exigences}} \\ \hline
  \multicolumn{1}{c|}{\textbf{FS 1}} &
    \begin{tabular}[c]{@{}l@{}}Doit être en mesure d'ouvrir plusieurs\\ micro-capsules dans un réacteur.\end{tabular} &
    \multicolumn{1}{c|}{\textbf{E 1}} &
    \begin{tabular}[c]{@{}l@{}}Ouverture jusqu'à 5 micro-capsules \\ par réacteur, sans contrainte sur \\ la présence de débris de verre.\end{tabular} \\ \hline
  \multicolumn{1}{c|}{\textbf{FS 2}} &
    \begin{tabular}[c]{@{}l@{}}Doit effectuer la fonction sur tout\\ les réacteurs de la plaque para-dox.\end{tabular} &
    \multicolumn{1}{c|}{\textbf{E 2}} &
    \begin{tabular}[c]{@{}l@{}}Répétabilité de la tâche\\ 48 fois par plaque.\end{tabular} \\ \hline
  \multicolumn{1}{c|}{\textbf{FS 3}} &
    \begin{tabular}[c]{@{}l@{}}Doit s'assurer de la libération\\ du réactif lors des essais.\end{tabular} &
    \multicolumn{1}{c|}{\textbf{E 3}} &
    \begin{tabular}[c]{@{}l@{}}La masse de réactif libéré est \\ précise à 0.01 mg.\end{tabular}
  \end{tabular}
  \caption{Fonctions de service}
  \label{tab:fctsservice}
  \end{table}

\subsubsection{Fonctions techniques}

Les fonctions techniques corresponde aux caractéristiques techniques que doit intégrer le produit.

\begin{table}[H]
  \centering
  \begin{tabular}{cl|cl}
  \multicolumn{2}{l|}{\textbf{Fonctions techniques}} &
    \multicolumn{2}{l}{\textbf{Exigences}} \\ \hline
  \multicolumn{1}{c|}{\textbf{FT 1}} &
    \begin{tabular}[c]{@{}l@{}}Doit fonctionner dans \\ un environnement contrôlé.\end{tabular} &
    \multicolumn{1}{c|}{\textbf{E 5}} &
    \begin{tabular}[c]{@{}l@{}}Glove box rempli uniquement d'azote \\ à température ambiante (environ 25 °C)\\ et en surpression \\(+15 Pa par rapport à 1 Atm).\end{tabular} \\ \hline
  \multicolumn{1}{c|}{\textbf{FT 2}} &
    Doit être dépannable facilement. &
    \multicolumn{1}{c|}{\textbf{E 6}} &
    \begin{tabular}[c]{@{}l@{}}Accessibilité simple et adapté \\ à un laborantin de chimie.\end{tabular} \\ \hline
  \multicolumn{1}{c|}{\textbf{FT 3}} &
    \begin{tabular}[c]{@{}l@{}}Doit alerter l'utilisateur en cas de\\ défaillance et éviter l'endommagement\\ des appareils.\end{tabular} &
    \multicolumn{1}{c|}{\textbf{E 7}} &
    \begin{tabular}[c]{@{}l@{}}Capteurs ou système de sécurité \\ en cas de défaillance \\ ou conditions anormale.\end{tabular} \\ \hline
  \multicolumn{1}{c|}{\textbf{FT 4}} &
    \begin{tabular}[c]{@{}l@{}}Doit assurer la sécurité de l'utilisateur \\ en cas de défaillance.\end{tabular} &
    \multicolumn{1}{c|}{\textbf{E 8}} &
    Protection contre les débris de verre.
  \end{tabular}
  \caption{Fonctions techniques}
  \label{tab:fctstechnique}
  \end{table}

\newpage
\subsubsection{Fonctions de contraintes}

Les fonctions de contraintes correspondent à des exigences imposé par le client ou par la configuration des lieux.

\begin{table}[H]
  \centering
  \begin{tabular}{ll|ll}
  \multicolumn{2}{l|}{\textbf{Fonctions de contrainte}} &
    \multicolumn{2}{l}{\textbf{Exigences}} \\ \hline
  \multicolumn{1}{c|}{\textbf{FC 1}} &
    \begin{tabular}[c]{@{}l@{}}Doit éviter la cross contamination\\ entre les réacteurs.\end{tabular} &
    \multicolumn{1}{c|}{\textbf{E 9}} &
    Système anti-projection. \\ \hline
  \multicolumn{1}{c|}{\textbf{FC 2}} &
    \begin{tabular}[c]{@{}l@{}}Doit s'assurer de la répétabilité \\ du système.\end{tabular} &
    \multicolumn{1}{c|}{\textbf{E 10}} &
    \begin{tabular}[c]{@{}l@{}}Sur 100 capsules, cassage\\ systématique.\end{tabular} \\ \hline
  \multicolumn{1}{l|}{\textbf{FC 3}} &
    \begin{tabular}[c]{@{}l@{}}Doit effectuer un cycle complet\\ dans un temps raisonnable.\end{tabular} &
    \multicolumn{1}{l|}{\textbf{E 11}} &
    Temps de cycle max 1h \\ \hline
  \multicolumn{1}{l|}{\textbf{FC 4}} &
    \begin{tabular}[c]{@{}l@{}}Doit s'assurer de l'intégrité des réactifs\\ après cassage.\end{tabular} &
    \multicolumn{1}{l|}{\textbf{E 12}} &
    \begin{tabular}[c]{@{}l@{}}Mesure effectuée durant les \\ phases de test.\end{tabular}
  \end{tabular}
  \caption{Fonctions de contrainte}
  \label{tab:fctscontr}
  \end{table}




\section{Catalogue des solutions et tests effectués}
\subsection{Pré-étude des micro-capsules}

Les micro-capsules sont fabriqué à partir de verre borosilicate, ce type 
de verre est souvent utilisé pour la verrerie de laboratoire. En effet, il comporte une 
très bonne résistance aux chocs thermique du faite de son faible coefficient de dilatation. 
De plus, il est résistant à de nombreux produits chimiques.
Néanmoins, son comportement mécanique peut s'avéré fragile en particulier pour de faibles
épaisseurs (voir \cite{report_borosilicate}).


\subsection{Liste des solutions envisagées}
\subsubsection{Canon à azote}
\paragraph{Principe de la solution}

\paragraph{Tests et simulations}

\paragraph{Conclusion}

\subsubsection{Implosion de la capsule}
\paragraph{Principe de la solution}

\paragraph{Tests et simulations}

\paragraph{Conclusion}


\subsubsection{Actionneur mécanique}
\paragraph{Principe de la solution}

\paragraph{Tests et simulations}

\paragraph{Conclusion}

\subsubsection{Fréquence de résonance}
\paragraph{Principe de la solution}

\paragraph{Tests et simulations}

\paragraph{Conclusion}


\subsection{Critères et choix de la solution}

\section{Tests des diverses solutions}


\section{Réalisation du prototype choisi}



\section{Programmation et configuration du robot}



\section{Conclusion}
\subsection{Vérification des objectifs du cahier des charges}


\subsection{Améliorations potentielles du projet}



\section*{Signature}

\begin{flushright}
    Yverdon, {\today\par}
\end{flushright}


\begin{figure}[H]
    \centering
    \Large Arnaud Arpino\\
    \includegraphics[height=3cm]{Images/Signatures/signature Arnaud.png}

\end{figure}
\newpage
\section*{Remerciements}

Je souhaiterais remercier, les divers enseignants et collaborateurs de la HEIG-VD, pour leur soutien et leur aide à divers moments 
de mon travail de Bachelor.

En particulier:
\begin{itemize}[label=\textbullet]
    \item M. Burdet, professeur de matériaux, pour son aide sur la caractérisation du verre
borosilicate, et ses idées de tests.
    \item M. Bonhôte, professeur de résistance des matériaux,  pour son aide lors de la réalisation 
de simulation par éléments fini sur Ansys.
\end{itemize}

\vspace{0.3 cm}
Je tiens également à remercier M. Costanzo, professeur référent, pour son aide et ses conseils tout au long de ce travail de Bachelor.

\vspace{0.3 cm}
Enfin, je tiens à remercier toute l'équipe de Swiss Cat + pour leur 
accueil et leur aide lors de la réalisation de ce travail de Bachelor.
%\input{Chapitres/commandes.tex}

\newpage
\printbibliography % To generate the bibliography
\appendix
\includepdf[pages=12,pagecommand={\thispagestyle{empty}\vspace*{-4cm}\section{Propriété du borosilicate 3.3}\label{pdf:borosilicate}}]{pdfs/Propriete_borosilicate_3-3.pdf}
%\includepdf[pages=2]{pdfs/TP03 Protocole polissage manuel.pdf}


\end{document}